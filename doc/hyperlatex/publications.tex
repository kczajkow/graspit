\section{Publications and References}
\label{sec:publications}

Copyright notice: this material is presented to ensure timely
dissemination of scholarly and technical work. Copyright and all
rights therein are retained by authors or by other copyright
holders. All persons copying this information are expected to adhere
to the terms and constraints invoked by each author's copyright. In
most cases, these works may not be reposted without the explicit
permission of the copyright holder.

All of the publication below are available at
\xlink{http://grasping.cs.columbia.edu}{http://grasping.cs.columbia.edu}.

For an introduction to GraspIt!, the most complete overview of its
core features is in:
\begin{itemize}
\item Andrew Miller and Peter K. Allen. ``Graspit!: A Versatile
  Simulator for Robotic Grasping''. IEEE Robotics and Automation
  Magazine, V. 11, No.4, Dec. 2004, pp. 110-122.
\end{itemize}

We recommend starting with that paper for the best introduction to the
system. Most of the papers below address individual features of the
simulator, you can read those that are relevant to the particular
project you are working on. The list of publication is presented in
chronological order, from oldest to newest. For each publication, we
also provide a short description of the parts of GraspIt! that it is
most relevant for.

\begin{itemize}

\item Andrew T. Miller and Peter K. Allen. "Examples of 3D Grasp Quality
Computations". In Proceedings IEEE International Conference on
Robotics and Automation, Detroit, MI, pp. 1240-1246, May 1999.

Introductory theory on the grasp quality metrics used by
GraspIt!. Discussed topics such as the Grasp Wrench Space, L1 and LInf
norms, epsilon and volume quality metrics, etc.

\item Andrew T. Miller. "GraspIt!: A Versatile Simulator for Robotic
Grasping", Ph.D. Thesis, Department of Computer Science, Columbia
University, June 2001.

Extremely detailed presentation of the GraspIt! core. The presentation
is mostly from a theoretical and research standpoint, but also covers
a number of practical implementation issues.

\item Danica Kragic, Andrew T. Miller, Peter K. Allen. "Real-time tracking
meets online grasp planning". In Proceedings IEEE International
Conference on Robotics and Automation, Seoul, Republic of Korea,
pp. 2460-2465, May 2001.

Application of GraspIt! to execute a grasping task with a real
robot. A real-life object is tracked using a camera, its position is
replicated in GraspIt! where a grasp is planned using a virtual
Barrett hand. The grasp is then executed using a real Barrett hand.

\item Andrew T. Miller, Steffen Knoop, Peter K. Allen, Henrik
I. Christensen. "Automatic Grasp Planning Using Shape Primitives," In
Proceedings of the IEEE International Conference on Robotics and
Automation, pp. 1824-1829, September 2003.

Detailed discussion of the Primitive-based grasp planner.

\item Andrew T. Miller, Henrik I. Christensen. "Implementation of
Multi-rigid-body Dynamics within a Robotic Grasping Simulator" In
Proceedings of the IEEE International Conference on Robotics and
Automation, pp. 2262 - 2268, September 2003.

Presents the theoretical framework between the dynamics engine in
GraspIt!. Covers topics such as time step integration, formulation of
contact and joint constraints as Linear Complementarity constraints,
etc. Shows how the full Linear Complementarity problem is assembled and
solved at each time step of the dynamic engine. A must-read for
understanding GraspIt! dynamics.

\item Rafael Pelossof, Andrew Miller, Peter Allen and Tony Jebara. "An SVM
Learning Approach to Robotic Grasping". In IEEE Int. Conf. on Robotics
and Automation, New Orleans, April 29, 2004, pp. 3512-3518.

Proposed the use of GraspIt! to generate large amounts labeled
grasping data that can be used to apply machine learning
algorithms. This code is not included in the current GraspIt!
distribution.

\item Matei Ciocarlie, Claire Lackner and Peter Allen. "Soft finger model
with adaptive contact geometry for grasping and manipulation
tasks". In IEEE Symposium on Haptic Interfaces for Virtual Environment
and Teleoperator Systems, Tsukuba, JP, March 19-21, 2007.

Discusses the Soft Finger contact as implemented in GraspIt!, covering
the analytical surface approximation, soft finger grasp wrench space
and formulation as linear complementarity constraints.

\item Corey Goldfeder, Peter K. Allen, Claire Lackner, Raphael
Pelossof. "Grasp Planning via Decomposition Trees". In IEEE
Int. Conference on Robotics and Automation, April 13, 2007, Rome.

Proposes an automatic method of decomposing an object into primitives
(superquadrics) to fully automate the task of primitive-based grasp
planning. This code is not included in the current GraspIt!
distribution.

\item Matei Ciocarlie, Corey Goldfeder and Peter Allen. "Dimensionality
reduction for hand-independent dexterous robotic grasping". In IEEE /
RSJ Conference on Intelligent Robots and Systems (IROS) 2007, San
Diego, Oct. 29 - Nov. 2

Introduces the eigengrasp concept and grasp planning in eigengrasp
space as an optimization problem solved through Simulated
Annealing. This is the recommended starting point if you are
interested in eigengrasps.

\item Matei T. Ciocarlie and Peter K. Allen. "On-Line Interactive
Dexterous Grasping". In Eurohaptics 2008, Madrid, June 10-13, 2008.

Present an application of eigengrasp planning for on-line interaction
with a human user. This is the theory behind the OnLinePlanner class
included in the distribution.

\item Corey Goldfeder, Matei Ciocarlie, Hao Dang and Peter K. Allen. "The
Columbia Grasp Database". In IEEE Int. Conf. on Robotics and
Automation 2009, Kobe.

Shows how GraspIt! can be used to generate a huge database of labeled
grasp data, and how this database can be used for data-driven grasp
planning algorithms. The database is publicly available. We are
currently working on releasing all the interface code that you need
for using both GraspIt! and the Columbia Grasp Database together. We
hope that this feature will be available in the summer of 2009 at
latest.

\end{itemize}

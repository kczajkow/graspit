\subsection{Bodies}
\label{sec:bodies}

There are two main types of bodies that exist in a GraspIt! simulation
world: static bodies (also known as obstacles) and dynamic bodies
(such as robot links and objects). Static objects do not participate
in the dynamics, but provide collision surfaces for dynamic
bodies. Note that this difference mainly applies when the dynamic
engine is being used; otherwise, dynamic bodies can be used as static
bodies as well. This document describes what makes up a basic GraspIt!
body, and what a dynamic body adds to that definition.

Every GraspIt! body has the following data associated with it:

\begin{description}
\item[Geometry:] this describes the shape of the body. It is stored as
  an Inventor scene graph, a format similar to VRML. This structure
  can contain pure shape nodes such as cubes, spheres, cylinders, and
  cones, as well as sets of 2D polygons that define a
  surface. Although GraspIt! (through the Inventor scene graph) can
  display all of these geometry types, the collision detection system
  only works with triangles. When any body is imported to a GraspIt!
  world, it is faceted into triangles, and these are used for
  detecting collisions and finding contact points. The units in these
  files are assumed to be millimeters. Note that the origin of a
  body's coordinate system is the origin of its geometry, as loaded
  from a file or created by the user. This can be a tricky aspect, as
  the origin of a body's geometry is fairly arbitrary. To counter
  this, dynamic bodies also use the notion of center of mass,
  explained below. However, the origin of a body's coordinate system
  is always the origin of it's geometry.
\item[Material:] the material affects the amount of friction possible
  at contacts on this body. For each pair of materials we define a
  coefficient of friction.  When the dynamic engine is not used,
  GraspIt! uses a static coefficient of friction for all bodies. When
  not using dynamics, this coefficient only affects grasp quality
  computations, not the relative motion of the bodies. During the
  dynamic simulation, the coefficient of friction affects the relative
  motion of the bodies in contact. We also use a dynamic coefficient:
  if the relative speed at a contact point exceeds a threshold
  currently set at 1mm/sec, a kinetic coefficient of friction is used.
\item[Transform:] each body keeps track of the 3D position and
  orientation of its body frame relative to the GraspIt! world
  coordinate system.
\item[Name:] a body's name is currently derived from the its filename,
  except in the case robot links, which are named using the robot's
  name and their kinematic chain and link numbers.
\end{description}

Dynamic bodies also have the following properties:

\begin{description}
\item[Mass:] this is expressed in grams.
\item[Center of mass:] this is a 3D position expressed relative to the
  body coordinate system. There are two main uses for this. The first
  one is to provide a stable point for grasp quality computations to
  use as reference. The second one is to be used as a reference point
  for transformation between forces and torques in dynamics. It should
  be more stable than the origin of the body's coordinate system,
  which is arbitrary.
\item[Inertia tensor:] this is the standard 3x3 mass distribution
  matrix. It is expressed relative to a coordinate frame that is
  aligned with the coordinate system of the body, but positioned at
  the center of mass. When stored in a file, it is scaled by 1/mass so
  that changes to the mass can be made by changing only the mass value
  above.
\item[Dynamic state:] two values, q and v, store the current position
  and velocity of the body's center of mass relative to world
  coordinates. q is expressed as a 7x1 vector: the first three values
  are the position, and the last four are the rotation in quaternion
  form. v is expressed as a 6x1 vector: the first three values are the
  linear velocity of the body, and the last three are the rotational
  velocity. When the dynamics updates each body state, the body
  transform is also updated. If a body is moved in the static mode,
  the position value of the dynamic state is also updated.
\end{description}

\subsubsection{Body Files}

Currently, GraspIt! employs the following convention: a Body file is
just the file that stores the geometry of the body, either in Inventor
(.iv) or VRML (.wrl) format. The geometry part is read in straight by
the scene graph. However, the geometry file can be enhanced by adding
some GraspIt! specific information at the beginning. All of this
information must be placed as Inventor comments (lines starting with
the character \#) so that it does not affect the process of loading the
geometry. None of the GraspIt! specific information is required. If
all of it is missing, Inventor will read in the geometry from the file
and GraspIt! will assign default values to all of its parameters.

GraspIt! parameters follow the convention: 

\texttt{\#parameter\_name value}

Where the \# sign is needed to "comment out" the line from Inventor's
perspective. For the parameter\_name , we have the following
possibilities:

\begin{description}
\item[material] the value is a string identifying the name of the
  material. Possible values are: frictionless, plastic, metal, wood,
  stone, rubber. Example:

\texttt{\#material plastic}

\item[mass] the value is the mass of the object, expressed in
  grams. Example:

\texttt{\#mass 300}

\item[cog] stands for center of gravity. The value is composed of 3
  entries, showing the coordinates of the center of gravity relative
  to the origin of the geometry. Example:

\texttt{\#cog 10.0 0.0 100.0}

\item[inertia\_matrix] the value is composed of the 9 entries in the
  body's inertia matrix, on three separate lines. Example:

\texttt{
\#inertia\_matrix\\
\#4853.0     -1.1196 -6.5156\\
\#-1.1196    4853.0  47.542\\
\#-6.5156    0.0     2357.6
}
\end{description}

For more examples, see the body files included with the distribution,
in \texttt{\$GRASPIT/models/objects}.

In the future, we would like to separate the GraspIt! parameters
from the geometry, and get rid of this crude parsing method. We are
considering an XML-like language, and parser, for specifying any body
parameters inside a file. This might be implemented in the next
release of GraspIt!.


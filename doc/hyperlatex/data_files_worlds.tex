\subsection{Worlds}

In GraspIt!, robots and bodies populate a simulation world. This
document describes how these elements can be added or deleted from a
world and describes the format of a world (.wld) file, which stores
the current state of the world.

When GraspIt! begins the world is empty. The user may either load a
previously saved world by choosing File $\rightarrow$ Open, or populate the new
world. To import an obstacle (a static body) or an object (a dynamic
body), File $\rightarrow$ Import Obstacle or File $\rightarrow$ Import Object, and then
choose the Body file (see the previous section on bodies). Note that
any Body file (regardless of whether it's meant for a static or
dynamic body) can be loaded as an obstacle (GraspIt! will just ignore
the dynamic parameters). However, when a body file is imported as an
Object, GraspIt! will automatically instantiate it as a dynamic
body. It will also try to find the dynamic parameters in the body file
and, if it can not find them, assign default values. Be aware that the
default values occasionally have unpredictable results.

To import a robot, use File $\rightarrow$ Import Robot, open the correct robot
folder, and select the robot configuration (.cfg) file.

To delete a body, select it, and then press the <DELETE> key. To
remove a robot, first select the entire robot (by double-clicking one
of the links when the selection tool is active) and press the <DELETE>
key.

Note: newly imported bodies or robots always appear at the world
origin. You can move existing bodies out of the way before importing a
new one. If you do not, than the newly imported body will overlap with
an old one, and you will have to temporarily toggle collisions in
order to move one of them out of the way.

When the user selects "Save" in the file menu, GraspIt! saves the
current world state in a text world (.wld) file. In the future this
will be stored in an XML format. The first line of the file
(e.g. \texttt{\#GraspIt! version 2.0beta}) identifies this file as a
GraspIt!  file. Each block of lines following that describes on
element of the world. The end of a block is marked with a blank
line. Each block begins a line describing the type of element. This
must be one of: Obstacle, GraspableBody, Robot, Connection or
CameraFull.
\begin{description}
\item[Obstacle:] the next line identifies the file name associated
  with the body. The filename is relative to \texttt{\$GRASPIT}. Next
  line should be a material type, and finally there is a transform
  that locates the body with respect to the world coordinates. The
  transform is stored by using a "T" followed by 4 numbers surrounded
  by parenthesis to represent the body orientation as a quaternion,
  and 3 numbers surrounded by brackets to represent the body position.
\item[GraspableBody:] the block is the same except the material line
  is omitted (the material is found in the body file).
\item[Robot:] the second line after this keyword identifies the robot
  configuration file. The third line contains values for DOF in the
  robot. Note that this might mean a single number per DOF or more
  information, depending on the DOF type. The last line is the
  transform described above.
\item[Connection:] indicates a connection between two robots. The
  second line of the block contains 3 numbers. The first is robot
  index of the parent robot, starting at 0. The second number is the
  kinematic chain number on the parent robot that the other robot is
  attached to. The third number is the index of the child robot. The
  next line (can be empty) specifies a body that is optionally used as
  a mount piece between the two robots. The last line of the block is
  the constant offset transform between the last link of the parent's
  kinematic chain and the base link of the child robot.
\item[CameraFull:] specifies the position, orientation and focal point
  of the camera. The next line contains 8 numbers: the first 3 are the
  camera position in World coordinates, the next 4 are the camera
  orientation specified as a quaternion, and the last one is the focal
  distance of the camera.
\end{description}

For an example, take a look at the \texttt{barrettGlassDyn.wld} file
supplied with this GraspIt! distribution.

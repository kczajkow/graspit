\subsection{Robots}
\label{sec:robotfile}

A Robot is made up of multiple links, connected into kinematic
chains. A link is simply a dynamic body, as described above. A Robot
always has a base link (called "palm" for hands) and one or more
kinematic chains attached to it. Each chain is in turn made up of a
succession of links, connected by joints. In order to define a robot,
two things are needed: the Body files for all the links that are part
of the Robot, plus an overall Robot configuration file, which has all
the kinematic information and references the appropriate body files
for the links. Here we describe the structure of the Robot
configuration file.

Robot configuration files can seem daunting at first, and they are a
bit annoying to get used to. However, you can start from one of the
many robots that are included with this distribution, and use it as a
starting block for your own robot that you are trying to build. For
the moment, a Robot configuration file is a simple text file which
must be parsed in the strict order given below. In the future, we hope
to move to more flexible XML-like format. In Robot configuration file,
any line that begins with the \# character is considered a
comment. Blank lines are ignored.

A Robot configuration file has three sections. The first one has the
overall information about the Robot (type, number of kinematic chains,
number of DOF's, etc). The second part contains the information for
each kinematic chain. The third part contains some optional additional
information. Most Robot files included with this GraspIt! distribution
are also commented to show what the purpose of each line is.

In this example, we will walk through the file for the Robonaut
hand. The first item in the file tells GraspIt! whether this is a
generic robot or hand or a particular subclass of one of those. A hand
should use the "Hand" class type. In some cases, if a robot has
special features or its own inverse kinematics algorithm, it is
necessary to use a subclass of these generic types, such as "Barrett",
or "Puma560":

\begin{verbatim}
#Robot ClassType
Robonaut
\end{verbatim}

The next item is the filename for the palm link:

\begin{verbatim}
#Palm filename (considered a link)
right_palm.iv
\end{verbatim}

This is followed by information on the DOF's. First comes the number
of DOF's. Then, for each DOF, a line with the information for
initializing that DOF. Note that a DOF can be connected to one or more
joints in the kinematic chains, this information will be supplied
later in the Robot configuration file. For more details about DOF's
and joints in GraspIt!, see the Joint Coupling and Underactuated Hands
chapter.

For each DOF, the line with initialization information contains the following:
\begin{itemize}
\item a letter showing the DOF type. For the Robonaut hand, all DOF's
  are of the type "rigid", depicted by the letter "r". This is the
  most common type of DOF in GraspIt!. Unless you are building robots
  with coupled joints (multiple joints connected to a single DOF), you
  can always use this type of DOF.
\item the default velocity for that DOF during an autograsp
  operation. This is used to pre-define the "closing" motion of a
  hand. For anthropomorphic hands, these pre-defined directions tell
  GraspIt! how to move each DOF in order to "autograsp", or how to
  "close the hand". This generally means moving the DOF's of the MCP,
  PIP and DIP joints in the "flexing" direction, and no movement for
  the abduction - adduction DOFs.
\item the max force the DOF can apply to each joint it is connected
  to. The unit is N * 1.0e6 * mm for torques and N * 1.0e6 for forces.
\item the Kp and Kv coefficients for a PD force controller built into
  the DOF
\item the visual scale of the dragger that allows the user to control
  this DOF through the GraspIt! GUI.
\item a number of optional parameters, depending on the DOF type. For
  the "rigid" DOF, no more parameters are needed.
\end{itemize}

\begin{verbatim}
#number of DOFs
14

#for each DOF:
#type default_velocities_for_auto_grasp max_effort_in_N_or_Nm dragger_scale
r 0.0  5.0e+9 1.0e+10 1.0e+7 20
r 1.0  5.0e+9 1.0e+10 1.0e+7 20
r 1.0  5.0e+9 1.0e+10 1.0e+7 20
r 0.0  5.0e+9 1.0e+10 1.0e+7 20
r 1.0  5.0e+9 1.0e+10 1.0e+7 20
r 1.0  5.0e+9 1.0e+10 1.0e+7 20
r 0.0  5.0e+9 1.0e+10 1.0e+7 20
r 1.0  5.0e+9 1.0e+10 1.0e+7 20
r 1.0  5.0e+9 1.0e+10 1.0e+7 20
r 1.0  5.0e+9 1.0e+10 1.0e+7 20
r 1.0  5.0e+9 1.0e+10 1.0e+7 20
r 1.0  5.0e+9 1.0e+10 1.0e+7 20
r 1.0  5.0e+9 1.0e+10 1.0e+7 20
r 1.0  5.0e+9 1.0e+10 1.0e+7 20
\end{verbatim}

The next piece of information contained in the Robot config file is
the number of kinematic chains (also called "fingers" for hands):

\begin{verbatim}
#Number of fingers
5
\end{verbatim}

We then move on to the next section, which contains information for
 each of the kinematic chains. We will only show here one of the five
 chains in the Robonaut hand, all the other chains follow the same
 convention. The first entries are the number of joints and links in
 the chain. Note that these are not always equal, as you can have up
 to 3 joints between two links.

\begin{verbatim}
#----------------f1 - index --------------------
#number of joints
4

#number of links
4
\end{verbatim}

Next comes the transform from the origin of the palm (which is also
the origin of the robot) to the base of this chain, which is where the
first joint in the chain is places. Any chain starts with a joint.

\begin{verbatim}
#Transforms from palm origin to base of the finger
t 87.883873 -27.044726 -16.619068
R
0.997205 -0.074709 0.000000
0.074425 0.993411 -0.087156
0.006511 0.086912 0.996195
\end{verbatim}

The next block is dedicated to the joints in the chain: one line for
each joint, with the Denavit-Hartenberg parameters and some additional
joint information. For each joint, the following entries are present:
\begin{itemize}
\item the four D-H parameters, in the order theta d a alpha. Either
  theta or d must be linked to a DOF number, as shown below.
\item the next two values are the joint limits.
\item the next values are optional (not present in the Robonaut file)
  and can contain things like joint friction coefficients, spring
  stiffness etc.
\end{itemize}

\begin{verbatim}
#Joint Descriptions (1 joint per line)
#(joints are ordered from closest to palm outward)
#(d# indicates DOF that this joint is connected to)
#(linear equations are of the form: d#*k+c [no spaces!])
#theta	d	a	    alpha	min	max
d3     	0	8.0518	90 	    -20 10
d4+-5.9	0	45.72   0      -10 85
d5     	0	26.67   0       0 	85
d5+5.9 	0	0       0  	    0 	85
\end{verbatim}

Finally, we have the block that puts links and joints together. This
block consists of a list of connection types and links. The connection
type tells us how each link is connected to the one before. It can be
one of the following: "Revolute", "Prismatic", "Universal", "Ball", or
"Fixed". Depending on which one is used, the chain will take some of
the joints from the joint list above, and put them together to create
a connection of that type. For example:
\begin{itemize}
\item Revolute: the link is connected to the one before by through a
  single revolute joint 
\item Universal: the link is connected to the one before by two
  revolute joints, usually with perpendicular axes
\item Ball: the link is connected to the one before by three revolute
  joints, usually with perpendicular axes
\end{itemize}

The link entry is simply a Body file that that link can be loaded
from. This is followed by the \texttt{lastJointAxis} number which
tells GraspIt! the index of that last joint axis that affects that
position of that link. The joint axes are numbered from 0. We are
hoping to get away from this convoluted way of specifying connections
pretty soon.

\begin{verbatim}
#Link Descriptions (1 link per line)
#filename lastJoint
#(links are ordered from closest to palm outward)
#(lastJoint is the last joint in the chain which can affect this link)
Revolute
yoke2.iv 0
Revolute
pph3l.iv 1
Revolute
mph4.iv 2
Revolute
bph3.iv 3
\end{verbatim}

The last part of the Robot file contains optional information. This
 usually includes things such as Eigengrasp information, connection to
 a Flock of Birds sensor, etc. These are described in more detail in
 the dedicated chapters of this manual.
